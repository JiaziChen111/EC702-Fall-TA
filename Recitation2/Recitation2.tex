% Document Preamble
\documentclass[11pt]{beamer}

\usepackage{amsmath}
\usepackage{multimedia}
%\usepackage[activeacute,english]{babel}
%\usepackage[latin1]{inputenc}
%\usepackage{xcolor}
%\usepackage[percent]{overpic}
%\usepackage{dcolumn}
%\usepackage{booktabs}
%\usepackage[absolute,overlay]{textpos}
%\usepackage{tikz}
%\usepackage{amsfonts}
%\usepackage{amssymb}
%\usepackage{makeidx}

\usepackage{anyfontsize}
\usefonttheme[onlymath]{serif}  %  Use this for sans-serif text and proper tex math, in italic serif (thanks to Max Floetotto for this)
\usefonttheme{structuresmallcapsserif} % Titles will appear in Small Cap Serif

% -----------------------------------------
% Center the Frame Title
% -----------------------------------------

\setbeamertemplate{frametitle} {
\begin{centering}
\vspace{0.1in} \insertframetitle
\par
\end{centering}}
\setbeamercolor{frametitle}{fg=black!50!blue}
\setbeamercolor{title}{fg=black!50!blue}
\setbeamercolor{button}{bg=black!50!gray,fg=white}


% -----------------------------------------
% Number the slides but don't show the distracting total number of slides
% -----------------------------------------

\setbeamertemplate{footline}{\hspace{350pt}\insertframenumber \vspace{0.1in}}  %  Thanks to Jihee Kim for figuring out how to suppress the total number of slides

% -----------------------------------------
% Get rid of the irritating navigation bar
% -----------------------------------------

\setbeamertemplate{navigation symbols}{}


\begin{document}

\title{Discussion 2: \\ Differential and Difference Equations - Examples}
\author{Stefano Pica \\ Boston University}
\date{September 13, 2019}

%%%%%%%%%%%%%%%%%%%%%%%
%%%%%%%% FRAME %%%%%%%%
%%%%%%%%%%%%%%%%%%%%%%%

\begin{frame}
\titlepage
\end{frame}

%%%%%%%%%%%%%%%%%%%%%%%
%%%%%%%% FRAME %%%%%%%%
%%%%%%%%%%%%%%%%%%%%%%%

\begin{frame}
\frametitle{Introduction}
\begin{itemize}\itemsep2ex
	\item We left off with some basic theory of difference and differential equations
	\item Today: examples related to the Solow Growth Model.
	\item Reference: Acemoglu, Introduction to Modern Economic Growth
	\item If you find any typos in these or future slides, please send me an email so I can correct them
\end{itemize}
\end{frame}

%%%%%%%%%%%%%%%%%%%%%%%
%%%%%%%% FRAME %%%%%%%%
%%%%%%%%%%%%%%%%%%%%%%%

\begin{frame}
\frametitle{Outline}
\tableofcontents
\end{frame}

%%%%%%%%%%%%%%%%%%%%%%%%%%%%%%%
%%%%%%%%%%% SECTION %%%%%%%%%%%
%%%%%%%%%%%%%%%%%%%%%%%%%%%%%%%

\section{The Solow Growth Model, discrete time}

%%%%%%%%%%%%%%%%%%%%%%%
%%%%%%%% FRAME %%%%%%%%
%%%%%%%%%%%%%%%%%%%%%%%

\begin{frame}
\frametitle{The Solow Growth Model}
\begin{itemize}
	\item I will solve this model step by step at the whiteboard.
	\item Assumption 1: Continuity, Differentiability, Positive and Diminishing Marginal Products, and CRS of the Production Function
	\item Assumption 2: Inada Conditions.
	\item Law of motion of capital
\begin{equation*}
K_{t+1} = s F(K_t, A_t L_t) + (1-\delta) K_t
\end{equation*}
	\item Population grows at rate n, technology at rate g. Then
\begin{equation*}
k_{t+1} = \frac{1}{(1+n)(1+g)}[s f(k_t) + (1-\delta) k_t]
\end{equation*}
where $k=K/AL$ and $f(k)=F(k,1)$.
	\item In steady state $\frac{f(\bar{k})}{\bar{k}} = \frac{(1+n)(1+g) - (1-\delta)}{s}$.
\end{itemize}
\end{frame}

%%%%%%%%%%%%%%%%%%%%%%%%%%%%%%%
%%%%%%%%%%% SECTION %%%%%%%%%%%
%%%%%%%%%%%%%%%%%%%%%%%%%%%%%%%

\section{The Solow Growth Model, continuous time}

%%%%%%%%%%%%%%%%%%%%%%%
%%%%%%%% FRAME %%%%%%%%
%%%%%%%%%%%%%%%%%%%%%%%

\frame{\tableofcontents[currentsection,hideothersubsections,currentsubsection]}

%%%%%%%%%%%%%%%%%%%%%%%
%%%%%%%% FRAME %%%%%%%%
%%%%%%%%%%%%%%%%%%%%%%%

\begin{frame}
\frametitle{The Solow Growth Model}
\begin{itemize}\itemsep2ex
	\item Same assumptions as previous slide
	\item Assume also exponential population growth. The law of motion of capital becomes
\begin{equation*}
\dot{k(t)} = s f (k(t)) - (n+\delta) k(t)
\end{equation*}
	\item A steady state involves $k(t)$ remaining constant at some level $k^*$ for any $t$. It is unique and such that:
\begin{equation*}
\frac{f(k^*)}{k^*} = \frac{n+\delta}{s}
\end{equation*}
	\item In steady state, the amount of investment is used to replenish the capital.
\end{itemize}
\end{frame}

%%%%%%%%%%%%%%%%%%%%%%%
%%%%%%%% FRAME %%%%%%%%
%%%%%%%%%%%%%%%%%%%%%%%

\begin{frame}
\frametitle{The Law of Motion of Capital}
\begin{itemize}\itemsep2ex
	\item \textbf{Theorem}: Let $g:\mathbb{R} \rightarrow \mathbb{R}$ be a continuous function, suppose there exists a unique $x^*$ such that $g(x^*)=0$. Moreover, suppose $g(x)<0$ for all $x>x^*$ and $g(x)>0$ for all $x<x^*$. Then the steady state of the non-linear differential equation $\dot{x(t)} = g(x(t))$, $x^*$,  is globally asymptotically stable, i.e., starting with any  $x(0), x(t) \rightarrow x^*$.
	\item \textbf{Proposition}: Suppose Assumptions 1 and 2 hold, then the basic Solow growth model in continuous time with constant population growth and no technological change is globally asymptotically stable, and starting from any  $k(0)>0$, $k(t) \rightarrow k^*$.
	\item Graph on the board.
\end{itemize}
\end{frame}

%%%%%%%%%%%%%%%%%%%%%%%
%%%%%%%% FRAME %%%%%%%%
%%%%%%%%%%%%%%%%%%%%%%%

\begin{frame}
\frametitle{Dynamics with the Cobb Douglas Production Function}
\begin{itemize}\itemsep2ex
	\item As an example, consider the production function
\begin{equation*}
F[K,L,A] = A K^\alpha L^{1-\alpha}
\end{equation*}
	\item Per capita production function is $f(k)=Ak^\alpha$, and the law of motion is
\begin{equation*}
\dot{k(t)} = s A k(t)^\alpha - (n+\delta) k(t)
\end{equation*}
and in the steady state
\begin{equation*}
k^* = \left ( \frac{sA}{n+\delta} \right )^{\frac{1}{1-\alpha}}
\end{equation*}
\end{itemize}
\end{frame}

%%%%%%%%%%%%%%%%%%%%%%%
%%%%%%%% FRAME %%%%%%%%
%%%%%%%%%%%%%%%%%%%%%%%

\begin{frame}
\frametitle{Dynamics with the Cobb Douglas Production Function}
\begin{itemize}\itemsep2ex
	\item To solve the law of motion of capital, let $x(t)=k(t)^{1-\alpha}$, so as to write the law of motion as
\begin{equation*}
\dot{x(t)} = (1-\alpha) sA - (1-\alpha) (n+\delta)x(t)
\end{equation*}
	\item which is a liner differential equation, with a general solution
\begin{equation*}
x(t) = \frac{sA}{n+\delta} + \left [x(0) - \frac{sA}{n+\delta}  \right ]  \exp (- (1-\alpha)(n+\delta)t)
\end{equation*}
	\item and changing the variable back
\begin{equation*}
k(t) = \left \{\frac{sA}{n+\delta} + \left [k(0)^{1-\alpha} -  \frac{sA}{n+\delta}  \right ]  \exp (- (1-\alpha)(n+\delta)t) \right \}^{\frac{1}{1-\alpha}}
\end{equation*}
	\item Note stability: starting at any $k(0)$ capital will converge to its steady state value.
\end{itemize}
\end{frame}

%%%%%%%%%%%%%%%%%%%%%%%%%%%%%%%
%%%%%%%%%%% SECTION %%%%%%%%%%%
%%%%%%%%%%%%%%%%%%%%%%%%%%%%%%%

\section{Solow with two Types of Capital}

%%%%%%%%%%%%%%%%%%%%%%%
%%%%%%%% FRAME %%%%%%%%
%%%%%%%%%%%%%%%%%%%%%%%

\frame{\tableofcontents[currentsection,hideothersubsections,currentsubsection]}

%%%%%%%%%%%%%%%%%%%%%%%
%%%%%%%% FRAME %%%%%%%%
%%%%%%%%%%%%%%%%%%%%%%%

\begin{frame}
\frametitle{Solow with Two types of Capital}
\begin{itemize}\itemsep2ex
	\item Consider a Cobb Douglas production function using two types of capital: \textit{equipment} $K_e$ and \textit{structures} $K_s$
\begin{equation*}
Y(t) = K_e(t)^\alpha K_s(t)^\beta (A(t) L(t))^{1-\alpha-\beta}
\end{equation*}
	\item Assume constant population growth $n$ and constant rate of labor-augmenting technological progress $g$.
	\item Assumption 1 (on production function) and 2 (Inada) hold true for both types of capital.
	\item Define effective capital ratios as $k_e=K_e/AL$ and $k_s=K_s/AL$.
	\item Model from Acemoglu, Physical and Human Capital
\end{itemize}
\end{frame}

%%%%%%%%%%%%%%%%%%%%%%%
%%%%%%%% FRAME %%%%%%%%
%%%%%%%%%%%%%%%%%%%%%%%

\begin{frame}
\frametitle{Laws of Motion of Capital}
\begin{itemize}\itemsep2ex
	\item Then the laws of motion of capital are
\begin{equation}
\dot{k_e(t)} = s_{K_e}f(k_e(t), k_s(t))  - ( \delta_{K_e} + g + n) k_e(t)
\end{equation}
\begin{equation}
\dot{k_s(t)} = s_{K_s}f(k_e(t), k_s(t))  - ( \delta_{K_s} + g + n) k_s(t)
\end{equation}
where $f(k_e, k_s)=k_e^{\alpha} k_s^{\beta}$, $\alpha+\beta<1$
	\item This is a system of differential equations: the two types of capital are state variables.
	\item A steady state is now defined in term of a couple ($k^*_e,k^*_s$) which satisfies the following two equations
\begin{equation}
\label{keprime}
s_{K_e}f(k_e^*, k_s^*)  - ( \delta_{K_e} + g + n) k_e^* = 0
\end{equation}
\begin{equation}
\label{ksprime}
s_{K_s}f(k_e^*, k_s^*)  - ( \delta_{K_s} + g + n) k_s^* = 0
\end{equation}
where \eqref{keprime} is the locus $\dot{k_e(t)}=0$ and \eqref{ksprime} is the locus $\dot{k_s(t)}=0$.
\end{itemize}
\end{frame}

%%%%%%%%%%%%%%%%%%%%%%%
%%%%%%%% FRAME %%%%%%%%
%%%%%%%%%%%%%%%%%%%%%%%

\begin{frame}
\frametitle{Stability - Graphical Representation}
\begin{itemize}\itemsep2ex
	\item The steady state ($k^*_e,k^*_s$) is
\begin{equation*}
k^*_e = \left [ \left (\frac{s_{K_e}}{n + g + \delta_{K_e}} \right )^{1-\beta} \left (\frac{s_{K_s}}{n + g + \delta_{K_s}} \right )^{\beta}   \right ]^{\frac{1}{1-\alpha-\beta}}
\end{equation*}
\begin{equation*}
k^*_s = \left [ \left (\frac{s_{K_e}}{n + g + \delta_{K_e}} \right )^{\alpha} \left (\frac{s_{K_s}}{n + g + \delta_{K_s}} \right )^{1-\alpha}   \right ]^{\frac{1}{1-\alpha-\beta}}
\end{equation*}
	\item In the $(k_e(t),k_s(t))$ space, the two loci -  \eqref{keprime} and \eqref{ksprime} - are upward sloping and intersect only once (not here, but can show by total differentiation, using concavity of the production function and Inada)
	\item Also, the steady state is globally stable. Diagrammatic proof at the board.
\end{itemize}
\end{frame}

%%%%%%%%%%%%%%%%%%%%%%%
%%%%%%%% FRAME %%%%%%%%
%%%%%%%%%%%%%%%%%%%%%%%

\begin{frame}
\frametitle{Stability - Analytical Derivation}
\begin{itemize}\itemsep2ex
	\item Recall the non-linear system of differential equations
\begin{equation*}
\dot{k_e} = s_{K_e}f(k_e, k_s)  - ( \delta_{K_e} + g + n) k_e
\end{equation*}
\begin{equation*}
\dot{k_s} = s_{K_s}f(k_e, k_s)  - ( \delta_{K_s} + g + n) k_s
\end{equation*}
	\item The linearized version around the steady state ($f_{k_e}$ and $f_{k_s}$ evaluated at the steady state)
\begin{equation*}
\dot{k_e} = [ s_{K_e}f_{k_e}  - ( \delta_{K_e} + g + n) ] (k_e - k_e^*) + s_{K_e}f_{k_s} (k_s - k_s^*)
\end{equation*}
\begin{equation*}
\dot{k_s} = s_{K_s}f_{k_e} (k_e - k_e^*) + [ s_{K_s}f_{k_s}  - ( \delta_{K_s} + g + n) ] (k_s - k_s^*)
\end{equation*}
	\item Now the system is linear and we can apply what we have learnt in discussion 1
\end{itemize}
\end{frame}

%%%%%%%%%%%%%%%%%%%%%%%
%%%%%%%% FRAME %%%%%%%%
%%%%%%%%%%%%%%%%%%%%%%%

\begin{frame}
\frametitle{Stability - Analytical Derivation}
\begin{itemize}\itemsep2ex
	\item The matrix of this system is
\[
M=
  \begin{bmatrix}
    s_{K_e}f_{k_e}  - ( \delta_{K_e} + g + n) & s_{K_s}f_{k_s}  \\
    s_{K_s}f_{k_e} & s_{K_s}f_{k_s}  - ( \delta_{K_s} + g + n)
  \end{bmatrix}
\]
	\item \textbf{Lemma}: For a $2by2$ matrix A, all eigenvalues must have negative real part if $trace(A)<0$ and $determinant(A)>0$.
\begin{equation*}
Trace(M) = s_{K_e}f_{k_e} + s_{K_s}f_{k_s}  - ( \delta_{K_e} + g + n)  - ( \delta_{K_s} + g + n)
\end{equation*}
\begin{multline*}
Det(M) = ( \delta_{K_e} + g + n) ( \delta_{K_s} + g + n) - \\ - s_{K_e}f_{k_e}( \delta_{K_s} + g + n) - s_{K_s}f_{k_s} ( \delta_{K_e} + g + n)
\end{multline*}
\end{itemize}
\end{frame}

%%%%%%%%%%%%%%%%%%%%%%%
%%%%%%%% FRAME %%%%%%%%
%%%%%%%%%%%%%%%%%%%%%%%

\begin{frame}
\frametitle{Stability - Analytical Derivation}
\begin{itemize}\itemsep2ex
	\item Note that since $f(k_e, k_s)$ is concave, then $f(k_e, k_s)>k_e f_{k_e} + 	k_s f_{k_s}$. Then in steady state
\begin{equation*}
f_{k_e} < \frac{f(k_e^*, k_s^*)}{k_e^*} =  \frac{\delta_{K_e} + g + n}{s_{K_e}}
\end{equation*}
 and so
\begin{equation*}
s_{K_e}f_{k_e} < \delta_{K_e} + g + n
\end{equation*}
Similarly
\begin{equation*}
s_{K_s}f_{k_s} < \delta_{K_s} + g + n
\end{equation*}
	\item The previous two inequalities show that $Trace(M)<0$.
\end{itemize}
\end{frame}

%%%%%%%%%%%%%%%%%%%%%%%
%%%%%%%% FRAME %%%%%%%%
%%%%%%%%%%%%%%%%%%%%%%%

\begin{frame}
\frametitle{Stability - Analytical Derivation}
\begin{itemize}\itemsep2ex
	\item Let's now show the determinant is positive.
\begin{multline*}
Det(M) = ( \delta_{K_e} + g + n) ( \delta_{K_s} + g + n) - \\ - s_{K_e}f_{k_e}( \delta_{K_s} + g + n) - s_{K_s}f_{k_s} ( \delta_{K_e} + g + n)
\end{multline*}
or
\begin{equation*}
Det(M) = \frac{s_{K_e} f}{k_e^*} \frac{s_{K_s} f}{k_s^*} -  s_{K_e}f_{k_e}\frac{s_{K_s} f}{k_s^*} - s_{K_s}f_{k_s} \frac{s_{K_e} f}{k_e^*}
\end{equation*}
or
\begin{equation*}
Det(M) = \frac{s_{K_e} s_{K_s}}{k_e^* k_s^*}f [ f - k_e^* f_{k_e} - k_s^* f_{k_s} ]
\end{equation*}
	\item Hence $det(M)>0$, and we get global stability.
\end{itemize}
\end{frame}




































%%%%%%%%%%%%%%%%%%%%%
%%%%%%%% END %%%%%%%%
%%%%%%%%%%%%%%%%%%%%%

\end{document}
